%_____Zusammenfassung, Ausblick_________________________________
\chapter{Conclusion}
In this work, we compared different discretization methods for a LTI continuous time system. Specifically, we have investigated RK methods of different order and Gauss collocation methods that allow piecewise polynomial input functions within MPC. We have found that the prediction accuracy and closed-loop performance of MPC increased with increasing order of the RK method that is used for discretizing the continuous-time system model. Our simulation example also showed that we can achieve similar accuracy between RK1 with small sampling time and RK4 with large sampling time, however, this yields increased computational cost due to the increased number of optimization variables. For the Gauss collocation method, we investigated the state trajectory for different-order hold control input. It was shown that the control invariant set of FOH control input is larger than ZOH control input, as FOH can consider more input shapes than ZOH.


For future research and further extension, we  can do the same investigation on nonlinear systems, considering even higher-order hold control inputs beyond FOH, e.g., second-order hold. We also found that in Fig. \ref{fig:FOHINPUT} there are big jumps at each sampling instant, it would also be interesting to investigate how to add possible continuity constraint which would be to fix the end point of one interval with the initial point of the following interval. 
